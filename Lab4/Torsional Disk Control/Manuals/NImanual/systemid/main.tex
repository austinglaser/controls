
\begin{center}  \textbf{Objective}
\end{center}
We provide several LabVIEW VIs that can be used to quickly collect
and analyze data from a plant to determine model parameters.  This
section provides instructions for using the VIs and presents results
from several ECP torsional plant units.

\ejnote{I go over the plant model so that it is clear what the
  parameters we are determining are.  Perhaps it would be better to
  have a separate system model section?  Another idea is to create
  ``solutions'' for each lab and then present the model in the Lab 2
  solutions and the system identification stuff in the Lab 3 one. I
  think this would prevent a lot of duplication.}

\section{System Identification}

The system identification problem is to create an accurate dynamical
model of the system.  The first step is to define the equations of
motion for the plant.

The plant consists of the bottom, middle, and top disks with
rotational inertias $J_1$, $J_2$, and $J_3$ and angles $\theta_1$,
$\theta_2$, $\theta_3$. Each disk has a viscous dissipation force with
damping coefficients $c_1$, $c_2$, and $c_3$.  There are torsional
springs connecting the bottom and middle disks and the middle and top
disks with respective spring constants $k_1$, $k_2$.  The power
amplifier and motor are treated as a single gain, $k_{h}$, with no
dynamics.  The resulting state space model is


\begin{equation}
\dot{x} = 
\left[\begin{array}{c}
\dot{\theta_1} \\
\dot{\theta_2} \\
\dot{\theta_3} \\
\ddot{\theta_1} \\
\ddot{\theta_2} \\
\ddot{\theta_3} \\
\end{array}\right]
= 
\left[\begin{array}{cccccc}
     0           &     0                &      0           &     1            &     0            &     0            \\
     0           &     0                &      0           &     0            &     1            &     0            \\
     0           &     0                &      0           &     0            &     0            &     1            \\
-\frac{k_1}{J_1} &  \frac{k_1}{J_1}     &      0           & -\frac{c_1}{J_1} &     0            &     0            \\
 \frac{k_1}{J_2} & -\frac{k_1+k_2}{J_2} &  \frac{k_2}{J_2} &     0            & -\frac{c_2}{J_2} &     0            \\
     0           &  \frac{k_2}{J_3}     & -\frac{k_2}{J_2} &     0            &     0            & -\frac{c_3}{J_3} \\
\end{array}\right]
\left[\begin{array}{c}
\theta_1 \\
\theta_2 \\
\theta_3 \\
\dot{\theta_1} \\
\dot{\theta_2} \\
\dot{\theta_3} \\
\end{array}\right]
+
\left[\begin{array}{c}
0 \\
0 \\
0 \\
\frac{k_{h}}{J_1} \\
0 \\
0 \\
\end{array}\right]u
\end{equation}
\begin{equation}
y = \left[\begin{array}{cccccc} 
1 & 0 & 0 & 0 & 0 & 0 \\
0 & 1 & 0 & 0 & 0 & 0 \\
0 & 0 & 1 & 0 & 0 & 0 \\
\end{array}\right]x
\end{equation}

To complete the model, the values of the model parameters must be
determined.  

\subsection{Rotational Inertia}
The inertia parameters can be calculated from known quantities.  The
ECP manual provides the rotational inertia of an unloaded disk (0.0019
kg-m$^2$), the mass of each weight (0.5 kg), and the radius of a
weight (2.5 cm).  By treating the weights as cylinders, the rotational
inertia of a disk loaded with $N$ weights can be found:
\begin{equation}
J = J_{disk} + \sum_{i=1}^N \left(  \frac{1}{2}m r^2 + m R_i^2 \right)
\end{equation}
where $r$ is the radius of a weight and $R_i$ is the distance from the
weight to the center of the disk.  For the bottom disk, rotational
inertia of the motor-drive assembly can be included using the value from the
ECP manual (0.0005 kg-m$^2$).

\subsection{Spring and Damper Coefficients}

The spring and damper coefficients are most easily found by by clamping the
middle disk to create two independent second-order systems.  The
transfer function for each disk in this configuration is

\begin{equation}
\label{second_order_equality}
 \frac{Y(s)}{U(s)} = \frac{1/J}{s^2 + (c/J)s + k/J} = \frac{K}{s^2 +
 2\zeta \omega_n s + \omega_n^2} 
\end{equation}

where $\omega_n$, $\zeta$, and $K$ are the standard second order plant
parameters.  The impulse response of a second order system is an
exponentially decaying sinusoid given by the equation:

\[ y(t) = y_0 e^{-\sigma t} \sin \omega_d t \qquad \sigma = \zeta\omega_n \qquad \omega_d = \omega_n \sqrt{1-\zeta^2} \]

where $y_0$ is the initial angle of the disk.  

Measuring the impulse response is equivalent to choosing non-zero
initial conditions for the plant and measuring the unforced response
of the plant.  Any number of methods can be used to determine the
frequency, $\omega_d$, and the rate of decay, $\sigma$ from the data.
Once these parameters are known, the spring constant and damping ratio
of the corresponding disk can be calculated.  It is assumed that $c_3 = c_2$.

\subsection{Hardware Gain}
The hardware gain, $k_{hw}$ is the ratio of the applied torque and specified
voltage.  It represents the combined dynamics of the power amplifier
and motor, and is assumed to be constant for the lab course.

To find $k_{hw}$, the plant is reduced to a different second order
system by unclamping all disks and removing the weights from the top
and bottom disks.  For further accuracy, the top and bottom disks can
also be removed.  The resulting trasnfer function is 
\[ \frac{Y(s)}{U(s)} = \frac{k_{hw}/J}{s(s + C/J)} \]

To determine the gain, we apply an input $U(s)$ to the plant and
record the system response.  The same input is used to simulate the
plant by assuming $k_{hw}=1$ along with the parameters found earlier.
The actual gain is found by comparing the two responses:
\[ \frac{Y_m(s)}{Y_s(s)} = \frac{\frac{k_{hw}/J}{s(s + c/J)}U(s)}{\frac{1/J}{s(s + c/J)}U(s)} = k_{hw} \]
where $Y_m$ and $Y_s$ are the measured and simulated responses,
respectively.  Rather than compare the outputs for all time, the final
steady state values are used.

\section{VI Implementations}

Four LabVIEW VI's are provided for doing system identification:
\begin{description}
	\item[System ID Recorder] Records the unforced response to
	non-zero intial conditions to determine $C$ and $K$.
	\item[System ID Analyzer] Analyzes data from the system ID recorder.
	\item[Hardware Gain Recorder] Applies a pulse input to the
	system and records response to determine $k_{hw}$.
	\item[Hardware Gain Analyzer] Analyzes data from the hardware gain recorder.
\end{description}
	
The VI's were designed to conveniently and quickly capture and analyze
a large number of data sets.  They allow the user to quickly determine
parameters, compare simulated responses to actual data, and
investigate nonlinearities in the system.

\subsection{System ID Recorder}
    The System ID Recorder VI is implemented in
    \texttt{system\_id\_logger.vi}.  This VI records the unforced
    response of a disk and saves the results to a file. To use the VI,
    follow these steps:


\begin{enumerate}
   \item Disconnect the motor power or turn off the power amplifier.
   \item Securely clamp the middle disk to the frame. 
   \item Place weights on the disk to be used.  Calculate the disk
       inertia for the weight configuration.
   \item Select the appropriate disk from the pull-down menu on the VI
       front panel.
   \item Enter an appropriate recording time on the VI front panel.
      The time should be long enough for the disk to come to rest
      after an initial push, plus a little extra time for you to push
      the disk.
   \item Start the VI.  Wait until the FPGA is initialized and the VI
      begins plotting data.
   \item Gently move the disk and quickly release it.  The disk should
      not be turned more than 40 degrees or the spring could be
      damaged.
   \item When the VI finishes, a dialog box will ask you where to save
      the file.  Select a directory and enter a new filename, or
      select a file that was created using the same VI and plant
      configuration.  The VI will remember your choice and append
      future data to the file.  If you want to write to a new file,
      either clear the filename control on the front panel, or enter
      the new filename.
   \item You can run the VI repeatedly to gather more data from
   different initial conditions.  To use the data effectively with the
   analyzing VI, the same weight configuration should be used for all
   data runs saved in a file.  Create a new file if the number of
   weights are changed.

\end{enumerate}

Once you have recorded enough data, you can analyze it to find the
spring and damper coefficients using the analyzer VI.

\begin{center}\begin{tabular}{|p{0.75\textwidth}|}
\hline The data is saved to file using LabVIEW's \texttt{Write to
Spreadsheet} Sub-VI.  The file is an ASCII text file where each data
set consists of two rows.  The first is the time relative to the start
of the VI and the second is the measured position of the disk at
that time.  These files can easily be imported into Excel, MATLAB, or
another LabVIEW VI. \\
\hline
\end{tabular}\end{center}

\subsection{System ID Analyzer}

The System ID analyzer is implemented in
\texttt{system\_id\_analyzer.vi}.  When you run the VI, a dialog will
ask you choose the data file.  Only use files created with the System
ID Recorder. 
   
This VI uses recorded unforced responses and a calculated rotational
inertia to estimate the spring and damer coefficients for a disk.  The
parameters are plotted with respect to their data sets initial
conditions to provide some insight into non-linearities in the
system.  The VI will also simulate the system with the estimated
parameters for comparison against the recorded data.

\subsubsection{Impementation}
    
The VI first selects two peaks from the data.  The first peak is
chosen by taking the first peak after the global minimum of the data.
As long as the disk was only pushed to one direction to create the
initial conditions, this tends to select the first or second peak of
the unforced response.  This point $(t_0,y_0)$ is considered the
initial condition of the data.  The second point $(t_1, y_1)$ is the
last peak that is more than 5\% of the initial condition.  The damped
frequency and decay rate are calculated from these two points:

\begin{equation}
\omega_d = \frac{2\pi(N-1)}{t_1-t_0}
\qquad
\sigma = - \frac{\ln{y_0}-\ln{y_1}}{t_1 - t_0}
\end{equation}

where $N$ is the number of peaks between and including the two chosen
peaks.  These are converted to the damping ratio and natural frequency
of the system:

\begin{equation}
\omega_n = \sqrt{ \omega_d^2 + \sigma^2} \qquad
\zeta = \frac{\omega_n}{\sigma}
\end{equation}

The second order parameters are used to calculate the spring and
damper coefficients according to
Equ. \ref{second_order_equality}.

\begin{equation} 
c = 2 J \omega_n \zeta
\qquad
k = J \omega_n^2 
\end{equation}

\subsection{Hardware Gain Recorder}

The Hardware Gain Recorder is implemented in
\texttt{hardware\_gain\_logger.vi}.  This VI applies a 1 second wide
pulse to the plant and records the response.  Follow these steps to
use this VI:

\begin{enumerate}
    \item Make sure all disks are unclamped and the bottom disk is
       completely free to rotate.
    \item Remove all weights from the top and middle disk.  For
       additional accuracy, remove the disks themselves from the plant.
    \item Place weights on the bottom disk and calculate the disk
       inertia.
    \item Connect the power amplifier to the motor and turn on the
       power amplifier.
    \item Enter the desired pulse voltage on the VI front panel.
    \item Enter an appropriate recording time on the VI front panel.
       The time should be long enough for the system to come to rest.
    \item Run the VI.  After approximately 2 seconds, the pulse will
       be applied and the system will spin and slowly come to rest.  
    \item When the VI is finished, a dialog box will ask you where to
       save the file.  The filename will be saved and used for future
       data sets.  Use the same file for all data runs with the same
       weight configuration.  The recording time and voltage can
       change within the file.
    \item Once you've chosen a filename, you can use LabVIEW's ``Run
       continuously'' function to capture several data runs at a single
       value.
    \item Be sure to capture data for different pulse voltages to see
       how the gain changes with the applied voltage.
\end{enumerate}

\begin{center}\begin{tabular}{|p{0.75\textwidth}|}
\hline The data is saved to file using LabVIEW's \texttt{Write to
Spreadsheet} Sub-VI.  The file is an ASCII text file where each data
set consists of three rows.  The first is the time relative to the
start of the VI, the second is the measured position, and the third is
the input signal.  These files can easily be imported into Excel,
MATLAB, or another LabVIEW VI. \\ \hline
\end{tabular}\end{center}

\subsection{Hardware Gain Analyzer}

The hardware gain analyzer is implemented in
\texttt{hardware\_gain\_analyzer.vi}.  When you run the VI, a dialog
will ask you choose the data file.  Only use files created with the
Hardware Gain Recorder.
   
This VI uses recorded pulse responses to estimate the hardware gain of
the plant.  The hardware gain is plotted against the pulse voltage to
provide insight into non-linearities in the system.  The VI also
simulates the system with the estimated gain for comparison against
the actual data.


\subsubsection{Implementation}

The VI determines the voltage of the applied pulse and the final
steady state value of the system from the recorded data. A simulation
of the system is run using the specified rotational inertia and
damping coefficient with $k_{hw}=1$.  The hardware gain is determined as

\begin{equation}
k_{hw} = \frac{y_{exp}(t_f)}{y_{sim}(t_f)}
\end{equation}

where $t_f$ is the time of the last recorded measurement.