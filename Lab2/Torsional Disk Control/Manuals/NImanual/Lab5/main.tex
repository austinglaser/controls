\fancyhf{}
\fancyfoot[L]{\textbf{ECEN 4638--T.D. Murphey}}
\fancyfoot[R]{\textbf{Lab \#5}}
\fancyfoot[C]{\vspace{.2in}\thepage}

\chapter{Lab \#5:  Root Locus Design} 

\begin{center}  \textbf{Objective}
\end{center}

The purpose of this lab is to use the EPS Torsional Disk System (TDS) to learn
how to apply root locus techniques to a real, physical system.  You have already
seen how PID works on the system in the last lab.  Now, you are going to see
some of your experience validated analytically by analyzing the root locus for
the ECP system.

The lab will consist of three main components.  First, you will determine the
hardware gain that you think is appropriate for the system and what error bounds
are appropriate for the parameters we are using.  Second, you will determine the
root locus for the system given the ``collocated'' outputs (the outputs that are
located at the same place as the input) and the ``noncollocated'' outputs (the
outputs that are located somewhere other than the input location).  (This will
help you see why stabilizing the top disk is intrinsically harder than
stabilizing the bottom disk.)  Lastly, you will iteratively determine what
``good'' values of a PID controller are based on the root locus.


\vspace{.2in}

\begin{center} \textsc{Note that this lab will be a two-week lab.}
\end{center}

\begin{center} \textbf{Warning:} 
\end{center}

As usual, please always run the ECP unit with someone holding the power button,
in case something happens to go wrong.

\newpage
\section{Pre-Lab Tasks}

Read this \textbf{entire} document before starting the lab.  Aside from that, you may
wish to create a VI that will take the model of your system and plot a root
locus (see Section~\ref{sec-rootlocus}).  I strongly recommend that you use your
state-space representation for this purpose.  However, you may use either the
state-space or transfer function representations, as you see fit.

\section{Tasks}

\subsection{Task \#1--Making Sure Your Simulation Works}

The purpose of this part of the lab is to make sure that your simulation is
reasonably related to the real physical system.  

\noindent \textbf{Task:}  Using the hardware template, make sure that you are 
running the experimental system with a time steps of 0.01 s.

\noindent \textbf{Task:}  Run the experiment in one VI and  the simulation in 
another VI.  I would recommend a closed-loop step response of roughly
$\frac{\pi}{2}$ rad and an open-loop pulse of 0.3 Nm for 0.5 s would probably be a
good combination.  Make sure that your simulation and the experiment
have roughly the same behavior.  If necessary adjust the hardware gain
block to make your simulation match the experimental behavior as
closely as possible.

\subsection{Task \#2--Root Loci for Disk 1 and Disk 3 Outputs}

\noindent \textbf{Task:} Draw a block diagram that represents the system with a
proportional controller.  Label all the blocks and all the lines that connect
the blocks.  Lastly, give the physical units for each line.  

\noindent \textbf{Task:} In LabVIEW, create the root locus for each output (disk 1  and
3).  Which system is ``more'' stable using proportional feedback?  Defend your
answer.  Test this hypothesis experimentally.  For each disk, experimentally
estimate the value of $K$ where the system destabilizes.  You may approach this
estimation problem in whatever way you find most convenient--just detail it in
your writeup.

\noindent \textbf{Task:} Now assume that you have a PD controller of the form
$C(s)=K(1+K_Ds)$ so that $K_D$ represents that \emph{ratio} of the damping to
the gain.  Assuming that $K_D$ varies from 0 to 1, plot the root locus for five
relevent ratios that you choose.  (Again, do this for both disk 1 and disk 3
outputs.)  What is the best choice from your perspective?  Defend your rational
of your choice and test it experimentally.  Explain your experimental results.

\noindent \textbf{Task:}  Based on the root locus of your choice of ratio, 
try to estimate what value of $K$ and $K_D$ would give a rise time of 1 second
and less than 30\% overshoot.  (Yes, again for for both disk 1 and disk 3 as
outputs.)  To do this, you may wish to use the ``Interactive Root Locus''
that allows you to see how the poles change as the gain $K$ varies.  Implement
your design in hardware.  How close do you get?

\noindent \textbf{Task:}  Based on the root locus for each output and your
choice of $K$ and $K_D$, can you add an integral term $K_I$ to your PID
controller in order to get rid of steady-state error?

\subsection{Task \#3--Lead/Lag Controller}

\noindent \textbf{Task:}  We know that PID controllers can be approximated by
``Lead/Lag'' controllers.  How does your root locus analysis change if you
replace the PID controller with a Lead/Lag controller?  

\subsection{Task \#4--Root Loci for Uncertainties}

\noindent \textbf{Task:}  For your choice of $K$ and $K_D$ above, plot the root 
locus for two parameters you think introduce the largest amount of uncertainty.
For instance, you can plot the root locus with respect to $c_1$ and with respect
to $k_1$, if you think these add the most uncertainty.  You may find
\emph{Mathematica} or another symbolic toolbox helpful for this problem.

\noindent \textbf{Task:}  Based on the root locus, do these  parameters
represent a threat to the stability of the system (for either disk 1 or disk 3
as outputs)?

\noindent \textbf{Task:}  Based on the root locus, how do these parameters affect
the performance of the system?

\subsection{Task \#5--Design for and Difficulties with Disk 3}

\noindent \textbf{Task:}  Can you design a different compensator (perhaps with
imaginary poles) that makes the system stable with better performance?  Can you
improve upon your PID controller performance?  Try this first using the root
locus tool, then in simulation, and finally (if you are convinced you have a
good design) try it in hardware.  If you didn't get a design you liked, discuss
what the obstacles were and what you tried to use to overcome them.


\section{Useful Things to Know}

\begin{enumerate}
\item The analysis section of the LabVIEW manual on Moodle has been updated
to show you how to use LabVIEW's root locus tools.  

\item We highly recommend using the new ``Torsional Plant Interface
Template'' to access the FPGA.  The new template provides more information,
better error checking, and doesn't need the Local Variable Feedback
hack used in Lab 4.
\end{enumerate}

\vspace{0.2in}

\noindent Remember, if you get stuck on some part of the lab, ask your
classmates, the TA, or myself.


%% Local Variables:
%% TeX-master: "../LVmanual.tex"
%% End:
