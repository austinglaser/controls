\fancyhf{}
\fancyfoot[L]{\textbf{ECEN 4638--T.D. Murphey}}
\fancyfoot[R]{\textbf{Lab \#6}}
\fancyfoot[C]{\vspace{.2in}\thepage}

\chapter{Lab \#6: Frequency Domain Compensator Design}

\begin{center}  \textbf{Objective}
\end{center}

The purpose of this lab is to expose you to frequency domain design techniques.
In particular, you will use Bode plots and Nyquist plots to determine the
stability and robustness of a design.  Because you are now all largely familiar
with LabVIEW, the description of this lab will be kept rather brief.  However,
keep in mind that this is nevertheless a reasonably long lab\textendash don't put off the
work until the second week!




\vspace{.2in}

\begin{center} \textsc{Note that this lab will be a two-week lab.}
\end{center}

\begin{center} \textbf{Warning:} 
\end{center}

As usual, please always run the ECP unit with someone holding the power button,
in case something happens to go wrong.

\newpage
\section{Pre-Lab Tasks}

Read this entire document before starting the lab.  A significant part of this
lab can be completed without hardware, so I recommend you complete as much of it
as you can before coming to lab.
\section{Tasks}






\subsection{Task \#1--Lead Controller for Disk 1}

\noindent \textbf{Task:}  Design a Lead controller for Disk 1 as the output  
and plot its root locus and its Bode plot.

\noindent \textbf{Task:}  What is the associated gain margin (GM) and phase
margin (PM)?

\noindent \textbf{Task:}  How could you change your Lead controller to improve
your margins?  Can you do so while simultaneously increasing your overall system
gain?


\subsection{Task \#2--Controller Design for Disk 3}

\noindent \textbf{Task:}  What is difficult about using Disk 3 as the output
instead of Disk 1?  What happens in the root locus if you try and use a Lead
compensator in this case?

\noindent \textbf{Task:}  Plot the Bode plot for the open loop system with the
third disk as the output.  Is it stable?  How much gain $K$ will destabilize it,
if $K$ comes from a proportional feedback controller with unitary feedback?

\noindent \textbf{Task:}  Plot the Nyquist plot for the open loop system.  What
does this tell you about the system stability?

\noindent \textbf{Task:}  Print the Nyquist plot and Bode plot and draw the 
gain margin and the phase margin into both plots for a particular value of $K$
(a value close to what you used when you first stabilized this system with a
proportional controller).

\noindent \textbf{Task:}  Design a higher performance controller that allows you
to push the overall system gain up higher while not destabilizing.  You may find
the notion of a \emph{notch filter} to be useful here.  (This is described in
FPE, and the entire description there you may find useful.)  Can you improve
upon your PID controller performance?  In particular, make sure that you
maintain your gain and phase margin at the ``reasonable level'' you just
determined.  Test your design in simulation.  What do you think are limits on
the performance that this system can achieve?

\noindent \textbf{Task:}  
At this stage, the SISO tool in MATLAB (command: \emph{sisotool}) will probably be very
handy--you will probably want to take MathScript code and copy it to MATLAB.
Use the SISO Tool in MATLAB to improve your controller design from the previous
section.  When moving poles and zeros around, \emph{only} move the poles and
zeros of your controller, not the plant itself!

\noindent \textbf{Task:}  What is the gain margin and phase margin for your
final controller design?


\noindent \textbf{Task:}  Test your design in experiment.  If there are
differences between the experiment and the simulation, explain them.  


\vspace{0.2in}
         

\noindent Remember, if you get stuck on some part of the lab, ask your
classmates, the TA, or myself.


%% Local Variables:
%% TeX-master: "../LVmanual.tex"
%% End:
