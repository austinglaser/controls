\section{Introduction}  

LabVIEW is a graphical programming language in principle capable of the same
utility that programming in C or C++ can provide.  Some capabilities of C++ are
more difficult to obtain, but for the purposes of control systems\textendash the focus of
this short tutorial\textendash LabVIEW is an exceptionally convenient programming
language.  This is for two primary reasons:
\begin{enumerate}
\item LabVIEW enables programming that mirrors that graphical analysis tools
  (such as block diagrams) that we use to analyze control systems;
\item LabVIEW seamlessly (well, at least ideally seamlessly) incorporates
  ``hardware-in-the-loop'' needs into code.  
\end{enumerate}
We will see both of these become evident over the course of this tutorial.  

\subsection{Background and Purpose}

This tutorial is being written as an accompaniament to the new control
laboratory course ECEN 4638 \cite{murphey-ace2006} at the University of
Colorado, based on student comments from previous laboratories.  Hence, this
tutorial is not intended to be exhaustive\textendash exhaustive tutorials have not been
very beneficial to students because they tend to be overwhelming.  Instead, this
tutorial aims to very specifically introduce students to LabVIEW in such a way
that they can use it \emph{in specific application to control systems} and so
that they may become comfortable both with LabVIEW in particular and graphical
programming in general, both of which are becoming standards in industry.
Moreover, this type of exposure is intended to encourage more student
investigation (both in LabVIEW, controls, and life in general) rather than
explicitly laying out the solutions to all anticipated problems.  (This learning
philosophy is along the lines of \cite{murphey-te2006}.)

\subsection{Philosophical relationship between LabVIEW and other programming
  languages.}

LabVIEW is to C (C++, Fortran, etc) as C is to assembly.  More to come here.

Data Flow, blocks that represent data manipulation, graphical, similar to block
diagrams, lines represent data types, 

Lastly, it is worth mentioning that LabVIEW is compiled, not interpreted.  Among
other things, this often means that when you are using new functionality, it
will require some time for it to compile all the code, particularly the code
that references hardware on your machine.  Be patient!

\subsection{Organization}

This tutorial will be organized according to the needs of an introductory
controls course.  The goal is \emph{not} to systematically introduce students to
all of LabVIEW's capabilities. It is the author's view that this is simply an
untenable goal leading to certain failure and confusion on the part of the
students.  Instead, we will focus on using simple examples, as they become
useful, to illustrate new tools being used within LabVIEW.  As the tutorial
proceeds, it is assumed that less and less explicit instruction will be
required, so some hints may simply lead the reader to look at a particular tool
on his own and merely indicate that it is indeed a useful tool.



 

%% Local Variables:
%% TeX-master: "../LVmanual.tex"
%% End:

