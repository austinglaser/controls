\pagestyle{fancy}
\fancyhf{}
\fancyfoot[L]{\textbf{ECEN 4638--T.D. Murphey}}
\fancyfoot[R]{\textbf{Lab \#4}}
\fancyfoot[C]{\vspace{.2in}\thepage}

\chapter{Lab \#4:  Open-Loop and Hardware-Based PID Control Design}
\date{}

\begin{center}  \textbf{Objective}
\end{center}

The purpose of this lab is to use EPS Torsional Disk System (TDS) to learn how
to apply some simple control techniques to a reasonably high-order, non-trivial
system.  As such, there will be no pre-lab simulation, although there will be some
simulation in your analysis of your results.  We are also attempting to create a
benchmark for how well one can do using ``hack'' techniques that involve no
analysis.

The lab will consist of two main components.  First, you will figure out what
open-loop control you can apply to get the ECP unit to rotate 180 degrees and
then rotate back 180 degrees with four weights on the bottom disk and two
weights on the middle and top disks. Then, based on experiment, you will design 
P, PD, and PID controllers to meet some specifications and do the same
experiment.  

Please always run the ECP unit with someone holding the power button, in case
something happens to go wrong.  

\vspace{.2in}

\begin{center} \textsc{Note that this lab will be a one-week lab.}
\end{center}

\begin{center} \textbf{Warning:} 
\end{center}

As before, please do not break the TDS systems.  If you have questions, or are
unsure of a next step, please let us know.  However, the most important thing is
to \emph{never run the system while it is clamped down}!  If you follow this one
rule, it is extremely unlikely (though sadly not impossible) that you will break
the system.


\newpage
\section{Pre-Lab Tasks}

If you wish, you may want to create the open-loop VI and PID VI's ahead of
time.  Aside from that, there are no pre-lab tasks (primarily because almost
everyone is preparing for the controls lecture midterm!).  

\section{Tasks}



\subsection{Task \#1--Open-Loop Control}

Attach four weights to the bottom disk, two disks to the middle disk, and two
disks to the top disk, all at $7 cm$ from the center.  

\noindent \textbf{Task:} Using trial and error (or
simulation if you like) get an open-loop controller that moves the bottom disk
from an initial orientation of $0$ to a final orientation of $\pi$ and back to 0
in 5 seconds with an error of 2 degrees ($=2\pi/180\  rad$).  Repeat this
experiment 10 times.  What is the average error of the outcome?

\noindent \textbf{Task:} Move the weights to $9 cm$ and perform the experiment
again using the same open loop control inputs.  How much more error do you get?
Change the open-loop control to meet the same specifications.  How much did it
change?


\subsection{Task \#2--P, PD, PID Control}

\noindent Before running these experiments, please let the TA or myself see your
LabVIEW code.  Also, be prepared for the fact that you \emph{may destabilize the
  system}, so don't be caught unaware!  This is because you are introducing
feedback into a system, and feedback can make an otherwise stable system
unstable.  

\noindent \textbf{Task:}  With the weights at their $7 cm$ location, implement a
proportional controller $C(s)=K$ (read the end of the lab to see how to do
this).  Try to get a rise time of $1 s$ and less than $30\%$ overshoot and
settling time of $3 s$ and $e_{ss} \leq 2 deg$ with respect to a step input.  Once
you have your system stabilized using your controller, \emph{gently} rotate the
disk.  It should resist your push in much the same way a spring does.

\noindent \textbf{Task:}  Move the weights to  $9 cm$.  How does your
performance change?

\noindent \textbf{Task:}  With the weights at their $7 cm$ location, implement a
PD  controller ($C(s)=K+K_D s$) to meet the specifications.

\noindent \textbf{Task:}  Move the weights to  $9 cm$.  How does your
performance change?


\noindent \textbf{Task:}  With the weights at their $7 cm$ location, implement 
a PID controller ($C(s)=K+K_D s+ K_I \frac{1}{s}$) to meet the specifications.

\noindent \textbf{Task:}  Move the weights to  $9 cm$.  How does your
performance change?

\noindent \textbf{Task:}  Change the output of the system to be the top disk.
\emph{Be careful here} because you might destabilize the system.  Does this make
the control design easier or harder?  How does your controller for the bottom
disk perform for the top disk?  What are the qualitative differences?  How do
you explain the differences?  What values of $K$, $K_D$ and $K_I$ in $C(s)=K+K_D
s+ K_I \frac{1}{s}$ stabilize the system?



\subsection{Task \#3--Simulation}

\noindent \textbf{Task:}  Simulate your controller designs with the parameters
you found in Lab 3 and with the simulations you built in Lab 2.  Compare the
output of the actual physical system to the simulation.  How should one compare
two different responses? Relative to such a comparison, how close is the
simulation to the real system?  What parameters need to be fixed?


\noindent \textbf{Task:}  In simulation, can you improve your PID design (for
either output)?  

\noindent \textbf{Task:}  At some point in this lab, you have almost certainly
destabilized the system by making some parameter too large.  Use the
\emph{rlocus} command or the root locus tool to find the root locus of for the
system relative to this parameter.  Can you explain the instability you saw
using by using the root locus?





\section{Things You May Want To Know}

Here are some LabVIEW hints for this lab.

\begin{enumerate}
\item Use a Create Control command for the reference input--this will allow you
  to play the with controller ``on the fly.''
\item In order to create a feedback loop, you must use a local variable in
  LabVIEW.  (This is to avoid algebraic loops in the code.)  To do this, use the
  following procedure:
  \begin{enumerate}
  \item Instead of creating a wire that goes from the output to the summation
    block (that create your negative feedback), right click on the ``negative''
    portion of the summation block and ``Create Control.''  Name this control
    ``Feedback.''  
  \item Use the palette function \emph{All Functions--Structures--Local} to create a
    local variable.  Make sure it is in ``write'' mode, and wire the output that
    you are interested in into it.  Click on it and select ``Feedback.''  You
    now have successfully ``closed the loop'' of your system.  
  \end{enumerate}
\item When you need to take a time derivative for the ``D'' part of your
  controller, do not use the ``s'' block in the Simulation Module.  Instead, use
  the palette function \emph{All Functions--Analyze--Point by Point--Time
    Domain--Derivative} to take derivatives.
\item Be very careful to make sure all your units are the same.
\end{enumerate}




\vspace{0.2in}
         

\noindent Remember, if you get stuck on some part of the lab, ask your
classmates, the TA, or myself.


%% Local Variables:
%% TeX-master: "../LVmanual.tex"
%% End:
