\fancyhf{}
\fancyfoot[L]{\textbf{ECEN 4638--T.D. Murphey}}
\fancyfoot[R]{\textbf{Final Project}}
\fancyfoot[C]{\vspace{.2in}\thepage}

\chapter{Final Project Guidelines: \emph{A Lab on Performance}}

\begin{center}  \textbf{Objective}
\end{center}

The purposes of this lab are to allow you to really be concerned about
performance in the last lab task you do.  The overall goals of the final project
are to:
\begin{enumerate}
\setlength{\itemsep}{-3pt}
\item make Disk 1 go from $0$ to $\pi$ and stabilize back to $0$ as quickly as
  possible,
\item accomplish the same task for  Disk 3,
\item you will do so without breaking the ECP unit,
\item and do so knowing that I get to move the weights on the ECP device
  (hence, introducing uncertainty into the system).
\end{enumerate}
This is a ``competition'' in the sense that I will rank your results, but in
general your ranking will not affect your grade.  The only thing that will
affect your grade is your lab report, which will be due \textbf{December 13}.
Rules regarding the final project:
\begin{enumerate}
\setlength{\itemsep}{-3pt}
\item you choose your own groups,
\item everyone will write an individual lab report,
\item everyone will use the same ECP unit, 
\item total time will be measured from the time the disk has moved $2$ degrees
  ($0.035$ rad) until it has settled to within $2$ degrees,  
\item you are permitted to run your code beforehand and do any system
  identification on the particular ECP unit we are going to use (I'll point it
  out in class),
\item for those of you with final conflicts, etcetera, you may do your final
  project outside of the scheduled time.  We will have to agree on a time for
  everyone who cannot do it on Friday, December 9, to do it on Monday, December
  12.  
\end{enumerate}
Your lab report should include all aspects of the design process, including:
\begin{enumerate}
\setlength{\itemsep}{-3pt}
\item  numerical simulations of the task,
\item  numerical analysis (e.g., root locus, bode, nyquist, sensitivity analyses), 
\item symbolic calculations (e.g., gain margin, phase margin, steady state
  error).
\end{enumerate}
Questions?  Ask!


%% Local Variables:
%% TeX-master: "../LVmanual.tex"
%% End:
